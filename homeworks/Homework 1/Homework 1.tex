%% !TEX program = pdflatex
%% !BIB program = bibtex
\documentclass[12pt]{article}

\usepackage{amsfonts,amsmath,amssymb,mathtools,marvosym}
\usepackage[left=2.5cm,right=2cm,top=2cm,bottom=2.5cm]{geometry}
\usepackage{indentfirst,setspace,multirow}
\usepackage{graphicx,xcolor,float,epstopdf}
\usepackage{enumerate}
\usepackage[bookmarks=true,breaklinks=true,colorlinks,linkcolor=blue,citecolor=blue,urlcolor=blue]{hyperref}

\usepackage{array}
\newcolumntype{P}[1]{>{\centering\arraybackslash}p{#1}}
\newcolumntype{M}[1]{>{\centering\arraybackslash}m{#1}}
\newcolumntype{N}[1]{>{\arraybackslash}m{#1}}

\onehalfspacing
%\doublespacing

\setlength{\parindent}{0.5cm}
\setlength{\parskip}{0cm}
%\renewcommand{\baselinestretch}{1.15} % 1.6 for double

% ==============================================================================

\title{\textbf{Homework 1}}
\author{ECON312 Time Series Analysis \\ Instructor: Narek Ohanyan}
\date{}

\begin{document}

\maketitle

% ------------------------------------------------------------------------------

\section*{Instructions}

\begin{itemize}
    \item The homework is due at due-time on \textbf{due-date}.
    \item Homeworks must be typeset in Latex and submitted (uploaded to the course page) in pdf format named \verb|HW1_Name_Surname.pdf|.
\end{itemize}

% ------------------------------------------------------------------------------

\section*{Assignment 1}

Consider an $ AR(p) $ process given by
\begin{align*}
    y_{t} = \phi_1 y_{t-1} + \phi_2 y_{t-2} + \ldots + \phi_p y_{t-p} + \varepsilon_{t} \qquad \varepsilon_{t} \sim WN(0, \sigma^2)
\end{align*}
with $ \sigma^2 > 0 $.

The impulse response function of an autoregressive process may be generated by setting $ \varepsilon_{0} = \sigma $ and $ \varepsilon_{t} = 0 $ for $ t > 0 $, and simulating the process for the required horizon $ H $. The resulting series is the impulse response function.

For the following exercises, set $ \sigma = 0.4 $ and $ H = 20 $.

\begin{enumerate}
    \item Generate the impulse response function for an $ AR(1) $ process for the following values of $ \phi_1 $: $ \phi_1 = 0.2 $, $ \phi_1 = 0.4 $, $ \phi_1 = 0.6 $, $ \phi_1 = 0.8 $, $ \phi_1 = 0.9 $ and plot them on the same graph. On a separate graph, plot the impulse response function for $ \phi_1 = -0.6 $ and $ \phi_1 = -0.9 $.
    \item Discuss the effect of $ \phi_1 $ on the impulse response function.
    \item Find example values of $ \phi_1 $ and $ \phi_2 $ for which the impulse response function of an AR(2) process is a) hump-shaped, b) oscillatory, c) explosive. Plot the impulse response functions for each case.
    \item Discuss the effects of $ \phi_1 $ and $ \phi_2 $ on the impulse response function.
\end{enumerate}

% ------------------------------------------------------------------------------

\section*{Assignment 2}

Consider an $ ARMA(1, 1) $ process given by
\begin{align*}
    y_{t} = \phi y_{t-1} + \varepsilon_{t} + \theta \varepsilon_{t-1} \qquad \varepsilon_{t} \sim WN(0, \sigma^2)
\end{align*}
with $ | \phi | < 1 $ and $ \sigma^2 > 0 $.

\begin{enumerate}
    \item Find a representation of $ y_{t} $ in terms of $ \varepsilon_{t} $, $ \varepsilon_{t-1} $, $ \varepsilon_{t-2} $, \ldots
    \item Find the mean and variance of $ y_{t} $
    \item Find the first autocovariance of $ y_{t} $
    \item Find the first autocorrelation of $ y_{t} $
\end{enumerate}

% ==============================================================================

\end{document}