%% !TEX program = pdflatex
%% !BIB program = bibtex
\documentclass[12pt]{article}

\usepackage{amsfonts,amsmath,amssymb,mathtools,marvosym}
\usepackage[left=2.5cm,right=2cm,top=2cm,bottom=2.5cm]{geometry}
\usepackage{indentfirst,setspace,multirow}
\usepackage{graphicx,xcolor,float,epstopdf}
\usepackage{enumerate}
\usepackage[bookmarks=true,colorlinks,linkcolor=blue,citecolor=blue,urlcolor={black},breaklinks=true]{hyperref}

\usepackage{array}
\newcolumntype{P}[1]{>{\centering\arraybackslash}p{#1}}
\newcolumntype{M}[1]{>{\centering\arraybackslash}m{#1}}
\newcolumntype{N}[1]{>{\arraybackslash}m{#1}}

\onehalfspacing
%\doublespacing

\setlength{\parindent}{0.5cm}
\setlength{\parskip}{0cm}
%\renewcommand{\baselinestretch}{1.15} % 1.6 for double

% =================================================================

\title{\textbf{Homework 2}}
\author{ECON312 Time Series Analysis \\ Instructor: Narek Ohanyan}
\date{}

\begin{document}

\maketitle

% -----------------------------------------------------------------

\section*{Instructions}

\begin{itemize}
    \item The homework is due at \textbf{due-time} on \textbf{due-date}. No late submissions will be accepted.
    \item Students are encouraged to submit answers typed in TeX format. Such submissions will be rewarded with a bonus of 2\% of the final grade.
    \item Homeworks must be submitted (uploaded to the course page) in pdf format named \textit{Name\_Surname.pdf}.
\end{itemize}

% -----------------------------------------------------------------

\section*{Assignment 1}

Consider the following AR(1) model
\begin{align*}
    y_{t} & = c + \phi y_{t-1} + e_{t}
\end{align*}
where the error terms are themselves AR(1) processes
\begin{align*}
    e_{t} & = \rho e_{t-1} + u_{t} \qquad\qquad\qquad u_{t} \sim IID(0,\sigma^2).
\end{align*}

\begin{enumerate}
    \item Show that the process may be written in the form of an augmented Dickey-Fuller test equation
          \begin{align*}
              \Delta y_{t} = \alpha_0 + \gamma y_{t-1} + \alpha_1 \Delta y_{t-1} + u_{t}
          \end{align*}
          and conclude that autocorrelation in the error terms can be eliminated by augmenting the model with a lagged first difference of the dependent variable.
    \item What is the relationship between the parameters of the original model $ \left( c, \phi, \rho \right) $ and the parameters of the new model $ \left( \alpha_0, \gamma, \alpha_1 \right) $?
    \item Under what parameter values of $ \left( \phi, \rho \right) $ does the process become a unit root process (i.e. $ \gamma = 0 $)?
\end{enumerate}

% =================================================================

\end{document}